\documentclass[FM,BP]{tulthesis}
% tento dokument používá balíky specifické pro XeLaTeX a lze jej přeložit
% jen XeLaTeXem, nemáte-li instalována použitá (komerční) písma, změňte
% nebo vymažte příkazy \set...font na následujících řádcích

\newcommand{\verze}{1.10}

\usepackage{polyglossia}
\setdefaultlanguage{czech}
\usepackage{xevlna}

\usepackage{makeidx}
\makeindex

% fonty
\usepackage{fontspec}
\usepackage{xunicode}
\usepackage{xltxtra}
% \setmainfont[Mapping=tex-text,BoldFont={* Bold},Numbers=OldStyle]{Baskerville 10 Pro}
% \setsansfont[Mapping=tex-text,BoldFont={* Bold},Numbers=OldStyle]{Myriad Pro}
% \setmonofont[Scale=MatchLowercase]{Vida Mono 32 Pro}

% příkazy specifické pro tento dokument
\newcommand{\argument}[1]{{\ttfamily\color{\tulcolor}#1}}
\newcommand{\argumentindex}[1]{\argument{#1}\index{#1}}
\newcommand{\prostredi}[1]{\argumentindex{#1}}
\newcommand{\prikazneindex}[1]{\argument{\textbackslash #1}}
\newcommand{\prikaz}[1]{\prikazneindex{#1}\index{#1@\textbackslash #1}}
\newenvironment{myquote}{\begin{list}{}{\setlength\leftmargin\parindent}\item[]}{\end{list}}
\newenvironment{listing}{\begin{myquote}\color{\tulcolor}}{\end{myquote}}
\sloppy

% deklarace pro titulní stránku
\TULtitle{Rozpoznávání emocí v audio nahrávkách s využitím hlubokých neuronových sítí}{}
\TULauthor{Tomáš Petříček}

% pro bakalářské, diplomové a disertační práce
\TULprogramme{B2646}{}{}
\TULbranch{1802T007}{Informační technologie}{Information technology}
\TULsupervisor{Ing. Lukáš Matějů Ph.D.}
\TULyear{2021}

% Vložil Koprnický, použití bibLateXu
% BibLaTeX settings
\usepackage[ 
backend=biber
%,style=iso-authoryear
,style=iso-numeric
%,style=numeric
%,sortlocale=cs_CZ
,autolang=other
,bibencoding=UTF8
%,urldate=edtf
]{biblatex}
\addbibresource{citations.bib} % vložení seznamu literárních zdrojů v bib formátu / input of references in bib format

%%%%%%%%%%%%%%%%%%%%%%%%%%
% Formátování podle pokynů FZS, čárka mezi jmény a poslední jméno se spojkou „a“
% při volbě iso-authoryear, 
\DeclareDelimFormat{multinamedelim}{\addcomma\space}

\DeclareDelimFormat{finalnamedelim}{%
  \ifnumgreater{\value{liststop}}{2}{\finalandcomma}{}%
  \addspace\bibstring{and}\space}

\DeclareNameAlias{author}{family-given/given-family} 
% kulaté závorky kolem citačního záznamu použitím \parencite{}
% nebo použitím standardního příkazu \cite{} s touto redefinicí \let\cite\parencite
%%%%%%%%%%%%%%%%%%%%%%%%%%

% Hranaté závorky kolem čísel v seznamu literatury při použitém iso-numeric
\DeclareFieldFormat{labelnumberwidth}{\mkbibbrackets{#1}}

\usepackage{csquotes} %užití biblatexu hlasí warnings, důvodem může být použití českých uvozovek v citacích! / solving of problems with Czech quotations
\urlstyle{same} %sazba url odkazů stejným fontem jako ostatní text, řešení problémů v zalamování hypertextových odkazů v citacích / url in references seting into the same form as text 

\begin{document}

\ThesisStart{male}
%\ThesisStart{zadani-a-prohlaseni.pdf}

\begin{abstractCZ}
\end{abstractCZ}

\begin{keywordsCZ}
\end{keywordsCZ}

\vspace{2cm}

\begin{abstractEN}
\end{abstractEN}

\begin{keywordsEN}
\end{keywordsEN}

\clearpage

\begin{acknowledgement}
\end{acknowledgement}

\tableofcontents

\clearpage

\begin{abbrList}
\end{abbrList}

\chapter{Úvod}
Mluvení je nám nejpřirozenější forma komunikace a emoce nám pomáhají si lépe porozumět. Díky emocím můžeme svému okolí ukázat svůj vnitřní psychologický stav. Při používání jiných forem komunikace je těžší vyjádřit své emoce, přesto lidé našli způsoby, jak je do komunikace zapojit. S rozvojem chatovácích platforem byly vyvinuty smajlíky, které reprezentují zjednodušený výraz lidského obličeje. Můžeme, tak ve zprávě poslat úplnější informaci a lépe si porozumět.\cite{DBLP:journals/speech/AkcayO20}

Nicméně i přesto, že jsme schopni vyjádřit své emoce, tak to neznamená, že se pochopíme. Emoce jsou subjektivní a každý z nás je může vnímat trochu jinak. Tato vlastnost emocí neusnadňuje ani vývoj systémů pro jejich rozpoznávání. Zatím nebyl vyvinut způsob, jak emoce měřit. Proto lidé usilují o vyvinutí systémů schopných rozpoznávat emoce bez explicitně zadaných instrukcí.

Modely strojového učení jsou schopné najít skryté vzory v datech a naučit se je rozdělovat. Pro učení modelu pro rozpoznávání emocí lze použít data získaná z textu, změn výrazu tváře, hlasu, gest nebo držení těla.\cite{konar_chakraborty_2015} Modely založené na rozpoznávání emocí z řeči mohou najít uplatnění na příklad při vývoji virtuálních asistentů. 

V posledních letech došlo k rozvoji osobních virtuálních asistentů jako jsou Siri a Alexa, které jsou používány pro hlasové ovládání elektronických zařízení. Mohou odesílat textové zprávy, přijímat telefonní hovory, přehrávat hudbu nebo vyhledávat ve webovém prohlížeči. Systém schopný rozpoznávat emoce může zlepšit komunikaci s asistentem tak, aby se nám zdála přirozenější.\cite{DBLP:journals/corr/abs-1912-10458} 

Dále může model najít uplatnění pro call centra. Data generovaná call centrem mohou posloužit k vývoji automatické obsluhy zákazníků nebo pro optimalizaci práce v call centru. Dispečerovi můžou být, podle emocionálního stavu zákazníka, nabídnuty scénáře podle, kterých je vhodné v dané situaci postupovat. Model může být využit také k vylepšení systémů doporučující videa nebo podcasty. Uplatnění může najít také při vývoji realističtějších her. \cite{konar_chakraborty_2015}

Systémy na rozpoznávání emocí nejsou v současnosti rozšířené, jsou zatím ve fázi vývoje. Modely jsou trénované z velké části na uměle vytvořených datových sadách a nejsou tak vystavěny datům z reálného života. \cite{konar_chakraborty_2015}

Cílem této práce je nashromáždit veřejně dostupné datové sady pro rozpoznávání emocí. Vybrat příznaky pro rozpoznávání. Vytvořit metodu pro poskytování dat do neuronové síti. Vytvořit model a uskutečnit několik pokusů během, kterých se budou měněny hyperparametry sítě. Následně výsledky prezentovat. Výsledky této práce mohou vést k vylepšení dosavadních technik pro rozpoznávání emocí z řeči. 

\chapter{Teoretická část}

\subsection{Emoce}
V současnosti není zvolena jednotná definice emocí, existuje jich mnoho. Emoce popisují náš vnitřní stav a jejich tvorba je ovlivněna mnoha faktory jako je osobní zkušenost, fyzické, jednací a komunikační reakce. Pro úlohu rozpoznávání emocí je důležité vědět, jak lze emoce rozdělit\cite{DBLP:journals/speech/AkcayO20}.

Emoce můžeme dělit 2 způsoby podle diskrétního modelu nebo prostorového modelu. Diskrétní model rozšiřuje emoce do kategorií. Mezi hlavní kategorie patří: smutek, radost, strach, hněv, znechucení a překvapení. Prostorový model dělí emoce do jednotlivých prostorů jako jsou mocenství, vzrušení nebo vliv. Vzrušení udává sílu emoce a má rozsah od znudění k nadšení. Výhodou diskrétního modelu je, že je na rozdíl od prostorového modelu intuitivnější.\cite{DBLP:journals/speech/AkcayO20}

\subsection{Datové sady}
K rozpoznávání emocí se používají datové sady, které musí být označkovány. Datové sady pro rozpoznávání emocí jsou děleny do tří hlavních kategorií podle toho, jak byla data získána. Data mohla být pořízena předstíráním, kdy herec při nahrávání předstírá, že emoci prožívá. Tento typ datové sady lze získat spoluprací s profesionálními herci nebo z video záznamů filmů a seriálů. Datovou sadu při spolupráci s herci lze relativně snadno sestavit, protože tvůrci mají poměrně velkou kontrolu nad celým procesem. Tímto způsobem získaná data, ale nemusejí odpovídat reálné situaci, a proto existují další 2 způsoby získávání datových sad. Při získávání vybuzených datových sad je mluvčí umístěn do situace, která se velice podobá reálnému životu. Situace jsou většinou vybírány tak, aby odpovídali potenciálnímu použití\cite{konar_chakraborty_2015}. Poslední způsob získávání dat pro rozpoznávání je z přirozené řeči. Data mohou být získána na příklad z rozhovorů z radií, televizních show nebo záznamů z call center. I přesto, že by tato data měla být pro rozpoznávání nejvhodnější, tak je mnohem obtížnější z nich sestavit datovou sadu. Na data se mohou vztahovat právní nároky a je s nimi více práce při zpracování.\cite{DBLP:journals/speech/AkcayO20}

Datové sady se od sebe mohou lišit také tím, kdo je namluvil. Mluvčí mohou být různého věku a pohlaví a mluvit různými jazyky. Datové sady se také liší tvrzeními, která byla vyslovena. Počet a kategorie rozpoznávacích emocí mohou být také různé.\cite{DBLP:journals/speech/AkcayO20}

Mezi často používané a bezplatné datové sady patří Danish Emotional Speech (DES) database, Berlin Emotional Speech (BES) database, Speech Under Simulated and Actual Stress (SUSAS) database nebo eNTERFACE datová sada.\cite{konar_chakraborty_2015}

\subsection{Současné poznání}
% Popis přístupů - co se na to používalo, co se používá teď nejlepší úspěchy
% Používané datasety
Přestože existuje mnoho datových sad, tak většina z nich má do 1 hodiny délky a nahrávky jsou namluveny kolem 10 mluvčích. V rozpoznává řečí je typicky potřeba datová sada s několika sty hodin a větší různorodost mluvčích\cite{konar_chakraborty_2015}.

Dále není v rozpoznávání určeno, které emoce rozpoznávat a jestli pro rozpoznávání používat diskrétní nebo prostorový model. Nicméně při použití diskrétního modelu jsou vymezeny takzvané „big N" emoce, mezi než patří na příklad: hněv, strach, smutek nebo radost\cite{konar_chakraborty_2015}.

Bylo provedeno mnoho pokusů pro rozpoznávání za použití různých modelů, datových sad, příznaků na různých jazycích. Parthasarathy a
Tashev porovnali neuronové sítě typu DNNs, RNNs a 1D-CNN na čínských datových sadách. S 1D-CNN dosáhli výsledků 56\% procent. Kannan Venkataramanan a Haresh Rengaraj Rajamohan provedli několik pokusů na datové sadě RAVDESS. Použili různé příznaky pro rozpoznávání jako MFCC nebo Log Mel Spectrogramy a vyzkoušeli modely jako různé varianty CNN a HMM. Nejlepší model 2D CNN s globalním average poolingem dosáhl 70\% na validačních datech a 66\% testovacích datech při rozpoznávání emocí do 14 kategorií\cite{DBLP:journals/corr/abs-1912-10458}.

\subsection{Zpracování dat pro rozpoznávání}
Data jsou v typickém systému pro rozpoznávání emocí nejdříve předzpracovány. Z dat jsou nejprve vytaženy příznaky na nízké úrovni. Poté jsou data rozděleny na rámce, z kterých jsou vytaženy příznaky pro rozpoznávání. Dále mohou být na příznaky použity techniky pro snížení počtu příznaků jako je PCA nebo LDA.\cite{konar_chakraborty_2015}.

Rámcování bývá prvním krokem předzpracování dat pro rozpoznávání, kdy je zvukový signál rozdělen na menší časové kousky zvané rámce, které nabývají většinou rozsahu mezi 10-30 milisekundami. Často se jednotlivé rámce překrývají 30\% až 50\%, aby se zachoval vztah mezi jednotlivými rámci \cite{DBLP:journals/speech/AkcayO20}. Důvodem rámcování je, že se emoce v průběhu řeči mohou měnit a rozdělení na menší časové úseky zajistí, že se emoce na zůstane v rámci jednoho rámce stejná \cite{konar_chakraborty_2015}.

Po rámcování většinou přichází okénkování, kdy je na jednotlivé rámce použita okénkovácí funkce. Snižuje amplitudu signálu na jeho okrajích a tím snižuje úniky, ke kterým může dojít při použití Rychlé Fourierovi Transformace (FFT). Pro okénková lze použít Hammingovu okénkovací funkci\cite{DBLP:journals/speech/AkcayO20}.

Dále mohou být použity techniky pro detekci hlasové aktivity. S jejich pomocí může ze signálu odstranit tichá. Mohou být použity metody jako je zero crossing rate, short time energy nebo autokorelační metody. Metoda zero crossing rate, která udává míru přechodu signálu z kladných do záporných hodnot a naopak v rámci jednoho rámce. Hodnota ukazatele je nízká v místech řeči a vysoká v místech ostatních. Při použité metody short time energy dosáhne vysokých hodnot energie v hlasové části a nízkou hodnotu v částech ostatních. Techniky pro odstranění tichých míst v řeči mohou snížit počet dat a zvýšit jejich přínos pro učení\cite{DBLP:journals/speech/AkcayO20}.

Může být použita také normalizace, která zmírňuje rozdíly v řečí mezi mluvčími a zároveň zachová přenášenou informaci. Normalizace může být použita na více úrovních na úrovni jednotlivých rámců nebo na úrovni celé datové sady. Nejčastěji se pro normalizaci používá z-normalizace.\cite{DBLP:journals/speech/AkcayO20}

Dále mohou být použity techniky na odstranění šumu, který se může během nahrávání do nahrávky dostat. Mezi nejčastěji používané techniky patří Minimum mean square error (MMSE) a log-spectral amplitude MMSE.\cite{DBLP:journals/speech/AkcayO20}.

Následuje zvolení vhodných příznaků pro rozpoznávání, kdy jsou z mnoha příznaků vybraný pouze nejvhodnější. Je mnoho příznaků, ale nejsou určeny příznaky, které by se hodili přímo pro rozpoznávání emocí. Ze signálu lze získat jak souhrnné tak lokální příznaky pro rozpoznávání. Mezi souhrnné  příznaky patří na střední hodnota, směrodatná odchylka, minimální nebo maximální hodnota. Lokální příznaky lze získat z jednotlivých rámců a zastupují krátkodobé změny v signálu.\cite{DBLP:journals/speech/AkcayO20}.

Příznaky pro rozpoznávání emocí mohou být založeny na intonaci (výška tónu), intenzitě (energie, Teager), linear prediction cepstral coefficients (LPCCs), perceptual linear prediction (PLP), cepstrálních koeficientech (MFCC), spektru (Mel frequency bands (MFBs)) a další.\cite{konar_chakraborty_2015}.

Dále může na příznaky pro rozpoznávání dělit na příznaky založené na čase a příznaky založené na spektru. Mezi příznaky založené na čase patří krátkodobá energie signálu, nejvyšší amplituda nebo nejmenší energie. Je velice jednoduché tyto příznaky získat a jednoduchým způsobem reprezentují zvukové nahrávky. Na druhou stranu příznaky založené na spektru nám mohou přinést hlubší porozumění. Mezi příznaky založené na spektru patří Mel-Frequency Cepstral Coefficients, spectral centroid nebo chroma coefficients\cite{DBLP:journals/corr/abs-1912-10458}.
% Udělat výčet používaných příznaků - 1-3 věty o nich

\subsubsection{Příznaky MFCC}
% Příznaky jsou popsány podrobněji, protože jsou používané v praktické části
Technika získávání MFCC příznaků spočívá v okénkování signálu, použití Diskrétní Fourierovy transformace, použití Mel banky filtrů, logaritmizace a použití inverzní Diskrétní kosínovi transformace. Ve fázi před zvýraznění je posílena energie vysokých frekvencí signálu. 

Při okénkování je zvukový signál je nejprve rozdělen na jednotlivé posuvné rámce. Dále je signál podroben okénkování, kdy je amplituda signálu snížena na konci a na začátku rámce. Může k tomu být použito Hammingovo nebo Hanningovo okénko. Dále signál převeden pomocí Diskrétní Fourierovi transformace do frekvenční oblasti. Na výsledné spektrum aplikujeme Mel banku filtrů, která je založena na Mel stupnici. Mel stupnice zohledňuje vnímaní zvukových frekvencí člověkem. Lidé lépe rozlišují mezi nízkými frekvencemi než mezi vysokými. Dále je signál zlogaritmován a je provedena Diskrétní kosínova transformace. \cite{hui_2019}

\subsection{Způsoby rozpoznávání emocí}
Pro rozpoznávání emocí se používají klasifikátory nebo regresory, které spadají do oblasti strojového učení s učitelem.\cite{DBLP:journals/speech/AkcayO20}. Modely strojové učení s učitelem vyžadují označená data. Každý vzorek musí mít přiřazený štítek udávající třídu, do které patří. V případě klasifikace jsou to diskrétní štítky, které odpovídají jednotlivým emocím jako jsou na příklad hněv, radost nebo smutek. V případě regrese jsou to desetinné hodnoty, které označují stupeň mocenství, vzrušení nebo dominance většinou v rozsahu od -1 do +1\cite{konar_chakraborty_2015}.

Při klasifikaci na základě rámců, musí mít klasifikátor přístup k širšímu okolí rámců, aby tak zachytil kontext. Důvodem je, že emoce ovlivňují dlouhodobé charakteristiky řeči. Přesný počet rámců, který by se měl vzít není určen. Na příklad při použít neuronové sítě typu RNN lze pro zachycení kontextu použít kolem maximálně 10 ramců, jinak model začne trpět problémem mizejícího gradientu. Dalším způsobem, jak klasifikovat emoce je podle statických příznaků jako je maximum, minimum nebo časová délka\cite{konar_chakraborty_2015}.

Při trénování se používá často cross validace, kdy se trénovací a validační sada mění a je vytvořeno několik modelů, kterých výsledky jsou zprůměrovány a je tak dosaženo smysluplnějším výsledkům. U rozpoznávání emocí je používáno především, protože jsou datové sady po většině menšího rozsahu, a tak má rozdělení do jednotlivých sad větší vliv na výsledek. Jako metrika pro ohodnocení modelu se používá přesnost nebo vážená přesnost. Vážená přesnost zohledňuje případný jiný počet vzorků pro každou třídu. Přesnost udává pravděpodobnost, že daný vzorek patří do předpovězené třídy\cite{konar_chakraborty_2015}.

Modely strojového učení se na datech učí parametry, kterým říkáme váhy a biasy. Parametry se během fáze učení mění. Jedna iterace trénovaní se nazývá epocha. Během každé trénovací epochy jsou mezi sebou příznaky vzorku a váhy modelu vynásobeny a je k nim přičten bias. Na konci modelu stojí aktivační funkce, která v případě klasifikace přiřadí vzorku pravděpodobnosti příslušnosti do jednotlivých tříd. Každou epochu je zjištěna ztráta modelu, která je pomocí metody gradient descent minimalizována. Výsledkem minimalizace je úprava parametrů modelu.

Každý model má také své hyperparametry, které se používají ve fázi učení.\cite{leonel_2019} Hyperparametry mohou měnit strukturu modelu, na příklad u hlubokých neuronových sítí to může být počet a šířka skrytých vrstev nebo aktivační funkce ve skryté vrstvě modelu. Další hyperparametry ovlivňují samotné učení mezi ně patří na příklad míra učení, regularizace nebo počet trénovacích epoch. Míra učení udává míru změny parametrů modelu při jejich aktualizaci. Regularizace zabraňuje přeučení.

Každý model by měl být, co nejobecnější. To znamená, že by měl dosahovat podobných výsledků jak na trénovacích datech tak na jakýkoli jiných. Když model nedostatečně zobecňuje, tak říkáme, že se underffiting. Naopak když se model učí v přílišném detailu, tak říkáme, že se přeučuje. Data jsou proto rozdělena na sadu trénovací a testovací. Model je natrénován na sadě trénovací a otestován na sadě testovací. Dobrý model je úspěšný v podobné míře na obou datových sadách. Trénovací sadu můžeme rozdělit také na sadu trénovací a validační. Validační sada se používá při trénování modelu. Na výsledcích této sady můžeme poznat dříve, jestli se model přeučuje nebo naopak, jestli se učí dostatečně. Při testování na testovací nebo validační sadě se parametry modelu nemění. 

Další způsob, jak ohodnotit model, jsou různé metriky. Důležitá metrika, která se používá při hodnocení klasifikátoru, je přesnost udávající poměr mezi správně klasifikovanými vzorky a všemi vzorky. Pro vyhodnocení modelu lze také použít matici záměny, která na jedné ose udává správnou třídu a na druhé předpovězenou třídu. Díky tomu můžeme jednoduše zjistit, jaké třídy si mezi sebou klasifikátor zaměňuje.

Existuje mnoho typů modelů pro klasifikaci jako jsou na příklad logistická regrese, Support Vector Machine, Decision Trees nebo neuronové sítě. \cite{fumo_2017} Dělí se do hlavních 2 skupin podle toho, jestli umí naučit klasifikovat lineárně nebo nelineárně. 

\section{Neuronové sítě}
%Obecný popis. Dopředná a zpětná propagace. 
Neuronové sítě jsou tvořeny vrstvami, které dělíme na lineární a nelineární. Lineární vrstvy tvoří parametry sítě. Nelineární vrstvy tvoří aktivační funkce. Abychom mohli označit neuronovou síť za hlubokou neuronovou síť, musí mít 2 a více lineární vrstvy. První vrstvě modelu říkáme vstupní vrstva a poslední vrstvě říkáme výstupní vrstva. Všem vrstvám mezi vstupní a výstupní vrstvou říkáme vrstvy skryté. Každá lineární vrstva má svojí šířku, která určuje počet neuronů ve vrstvě.

Všechny vrstvy sítě musí být derivovatelné, aby se mohl pro aktualizaci parametrů sítě použít algoritmus gradient descent. Derivace pro jednotlivé vrstvy sítě se počítají pomocí řetízkového pravidla. Během dopředného průchodu se v lineární vrstvě vynásobí váhy s příchozími příznaky a přičte se k nim bias.

Ve skrytých vrstvách se nejčastěji pro klasifikaci používá aktivační funkce ReLU, která při dopředném průchodu propustí pouze kladná čísla a všechny záporná čísla vynuluje.

Pro klasifikaci do více tříd se používá aktivační funkce Softmax. Výsledkem průchodu touto funkcí je vektor, který obsahuje pravděpodobnosti příslušnosti vzorku do jednotlivých tříd. Součet vektor je tedy roven jedné.

% Vybrané koncepty
\subsection{Křížová entropie}
Jako hodnotící kritérium pro klasifikaci do více tříd lze použít křížovou entropii, která měří rozdíl mezi dvěma rozděleními pravděpodobnosti. Minimalizací křížové entropie minimalizujeme rozdíl mezi pravděpodobnostním rozdělením trénovacích dat a pravděpodobnostním rozdělením předpovídaných hodnot.\cite{brownlee_2020}

\subsection{ReLU}
ReLU neboli rectified linear activation function je aktivační funkce, jejíž výstupem je buď nula, pokud je vstup záporný nebo se hodnota vstupu nezmění, pokud je kladný. Neuronové sítě používající tuto funkci se většinou učí snadněji a dosáhnou lepších výsledků. Výhodou této funkce, je, že nepodléhá přesycení, kdy jsou velká čísla na příklad u funkce sigmoid změněna na 1 a velmi malá čísla 0. Důsledkem přesycení je, že jsou funkce jako sigmoid nebo tanh velice citlivé na hodnoty okolo jejich středu a méně na odlehlejší hodnoty. Následně se to projeví při trénovaná modelu, kdy může dojít k problému mizejícího gradientu u hlubších neuronových sítí\cite{brownlee_2020_ReLU}.

\subsection{Softmax}
Softmax je funkce, která přemění vektor čísel na vektor pravděpodobností, kde jsou pravděpodobnosti pro jednotlivé prvky úměrné velikosti všech prvků vektoru. Součet všech výsledných pravděpodobností je roven jedné. Funkce se používá při klasifikaci do více tříd a stojí na konci klasifikátoru. Každá výstupní pravděpodobnost odpovídá jednotlivé třídě datové sady\cite{brownlee_2020_Softmax}.

% používané architektury
% Markovi nepopisovat - nejsou neuronovky
\subsection{Konvoluční neuronové sítě}
Jsou ve velké míře používány pro práci s obrázky. Skládají se z konvolučních vrstev. Každá konvoluční vrstva obsahuje sadu filtrů, jejichž parametry se síť učí. \cite{DBLP:journals/corr/abs-1912-10458}

\subsection{Rekuretní neuronové sítě}
Rekurentní neuronové sítě (RNN) jsou především používány při řešení úloh s daty závislými na čase. Rekurentní neuronové sítě si pamatují skrytý stav, který je vypočítán ze vstupu a vstupuje společně s následujícím vzorkem do sítě. Síť typu RNN je schopna si pamatovat větší kontext a použít ho při předpovídání. Pokročilejším modelem je LSTM, který je schopný si poradit s problémem mizejícího a explodujícího gradientu. RNN lze také použít k rozpoznávání emocí, kdy jsou jednotlivé rámce společně s jejich okolí postupně posílány do sítě. Délka rámce s okolím by měla mít alespoň 250 milisekund, aby se byl model schopný naučit rozpoznávat emoce. \cite{DBLP:journals/corr/abs-1912-10458}

\chapter{Praktická část}
\section{Výběr datových sad}
Datové sady byly vybrány na zásadě přehledu z vědecké článku \cite{DBLP:journals/speech/AkcayO20}. Hlavním kritériem při výběru byla dostupnost datové sady, proto byly vybrány sady bezplatné a dostupné pro vědecké účely. Ze zbylých datových sad byly vybrány pouze 3 anglické datové sady a jedna italská.

\subsection{RAVDESS}
Zkratka RAVDESS znamená Ryerson Audio-Visual Database of Emotional Speech and Song a označuje anglickou datovou sadu obsahující nahrávky řeči a písní. Spolu s nahrávkami zvuku byly pořízeny i video záznamy mluvčích. Nahrávky byly namluveny 24 herci, z kterých bylo 12 žen a 12 mužů činící datovou sadu pohlavně vyrovnanou. Mluvčí mluvili severoamerickou angličtinou. Nahrávky řeči zachycují 8 emocí: klid, radost, smutek, hněv, strach, překvapení, znechucení a neutrální stav. Každý herec namluvil 2 tvrzení ve 2 úrovních emocionální intezity, běžnou a silnou, pro všechny emoce. Namluvené výrazy byly „Kids are talking by the door" a „Dogs are sitting by the door". Počet zvukových nahrávek řeči je celkově 1440. Každá nahrávka v datové sadě má přiřazený štítek, který udává druh nahrávky, druh emoce, emocionální intezitu, tvrzení, číslo opakování a herce, který nahrávku namluvil. Délky nhrávek se pohybují kolem 3 minut. Datové sada je dostupná buď na webových stránkách smartlaboratory.org nebo na Kaggle.com.\cite{smart_lab}

\subsection{SAVEE}
Zkratka SAVEE znamená Surrey Audio-Visual Expressed Emotion a označuje anglickou datovou sadu pro rozpoznávání emocí. Obsahuje 480 promluv a rozlišuje 7 emocí, mezi které patří hněv, znechucení, strach, radost, smutek, překvapení a neutrální stav. Byla namluvena 4 mužskými herci mluvící britskou angličtinou ve věku mezi 27 až 31 lety. Pro tvorbu datové sady bylo vybráno 15 vět z datové sady TIMIT. Kromě řeči byly při nahrávání zaznamenávány i pohyby v obličeji mluvčích, kteří na něm měli namalovány modré značky.\cite{savee}

\subsection{TESS}
Zkratka TESS znamená Toronto emotional speech set a označuje anglickou datovou sadu pro rozpoznávání emocí. Obsahuje 2800 promluv, které byly namluveny 2 herečkami ve věku 26 a 64 let. Rozlišuje 7 emocí mezi něž patří hněv, znechucení, strach, radost, překvapení, smutek a neutrální stav. Každá herečka namluvila 200 promluv pro všechny emoce. Promluva vždy začínala slovy "Say the word" a končila jedním z 200 vybraných slov.\cite{tess}

\subsection{EMOVO}
EMOVO je italská datová sada pro rozpoznávání emocí. Zvukové nahrávky byly vytvořeny 6 herci, z kterých 3 byli muži a 3 ženy. Každý herec vyslovil 14 vět pro každou emoci. Datová sada rozlišuje 7 emocí mezi něž patří znechucení, strach, hněv, radost, překvapení, smutek a neutrální stav.\cite{COSTANTINI14.591}

\section{Příprava dat}
Data byla před vstupem do neuronové sítě předzpracována. Nejprve byl sjednocen formát nahrávek. Nahrávky byly převedeny na vzorkovací frekvenci 16 kHz a byl zachován jeden zvukový kanál. Pro převod byl použit nástroj FFmpeg, který je přístupný přes rozhraní příkazové řádky. V Pythonu lze k němu přistoupit pomocí modulu subprocess. Data byla načtena, převedena do požadovaného formátu a uložena do nových souborů. Správnost převodu byla ověřena.

Dále byly nahrávky převedeny na MFCC příznaky. K převodu byl použit nástroj HTK, který je opět přístupný z příkazové řádky. Pro převod byl použit konfigurační soubor. Zvukové nahrávky byly načteny, převedeny a výsledné příznaky byly uloženy do binárních souborů. Struktura binárních souborů je definovaní HTK. V Pythonu byl napsán, na GitHubu zveřejněn, menší modul PyHTK, pomocí kterého lze data načíst. Správnost načtených příznaků byla také ověřena.

Následovalo sjednocení označení datových sad. Každá z datových měla jinak strukturované značení a obsahovalo různé informace. Proto byly pro každou datovou sadu vytvořeny načítače, které převedli značení do jednotného stavu. Byla zachována pouze informace o tom, do které třídy nahrávka patří. Všechny datové sady obsahovali 7 emocí, kromě datové sady RAVDESS, která obsahovala navíc klidou emoci. Klidná emoce byla převedena na neutrální stav. Tím bylo získáno 7 základních emocí pro rozpoznávání mezi něž patřil smutek, hněv, radost, překvapení, strach, znechucení a neutrální stav.

Z datových sad byly vytvořeny datové sady pro rozpoznávání. Byla na příklad vytvořena datová sada se všemi anglickými nahrávky a datová sada pouze s italskými nahrávkami. Datové sady mohli zůstat celé nebo mohli být rozděleny na trénovací, validační a testovací sadu. Mohli být škálované. Měnil se také počet rozpoznávacích emocí. Všechny datové sady byly rozděleny stratificky, což znamená, že do každé sady byly vzorky rozmístěny rovnoměrně, tak aby bylo zastoupení jednotlivých tříd rovnoměrné. Jednotlivé sady byly uloženy jako souvislé pole numpy a do textových souborů byly zapsány štítky a délky jednotlivých vzorků, aby se při opětovném načtení dalo rozlišovat mezi nahrávkami.

\section{Načítání dat}
Data uložená v polích numpy byla načtena pomocí upravených Dataloaderů. Data pro trénování byla načítána jednotlivých rámcích zatím, co data pro validaci byla načítána po celých vzorcích. Při trénování bylo možné vyhodnotit přesnost trénování pro každý rámec, zatím co při validaci bylo možné hodnotit nahrávky podle přesnosti předpovědi pro celou nahrávku. Bylo to možné také, protože se model při validaci neučí.

Data byla načítána do modelu i s okolím. Ke každému rámci byly přidány rámce s nejbližšího okolí. Data na krajích sady byla rozšířena o okraje, aby při vybírání krajních rámců nedošlo překročení velikosti pole. Okraje byly tvořeny kopie krajních rámců.

\section{Vytvoření neuronové sítě}
Pro vytvoření neuronové sítě byl zvolen framework PyTorch, který je vyvíjen společností Facebook. Dále bylo možné použít framework TensorFlow, který je vyvíjen společností Google. Oba mají velice dobré dokumentace a rozsáhlou komunitu programátorů. Sestavení modelu v TensorFlow je jednoduché. Je možné společně s ním použít framework Keras, který následuje syntax knihovny pro strojové učení scikit-learn. Vytvoření neuronové sítě za pomocí frameworku Keras je velice jednoduché, nicméně neposkytuje tak velkou flexibilitu jako framework PyTorch, kde si kód pro učení, validaci a vyhodnocení může programátor jednoduše napsat sám podle své potřeby. 

Model byl vytvořen pomocí modelu Sequential a byl typu Multi Layer Perceptron, skládal se splně propojených lineárních vrstev, mezi kterými byly jako aktivační funkce umístěny ReLU. Na konci sítě stála funkce Softmax. Jako kritérium pro trénování byla použita křížová entropie.

\chapter{Závěr}

\nocite{*}
\printbibliography[title={Použitá literatura}] % sazba seznamu citací
\addcontentsline{toc}{chapter}{Použitá literatura} % vložení nadpisu do obsahu

\end{document}
